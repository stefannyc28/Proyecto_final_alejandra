
\documentclass{beamer}
\usetheme{Madrid}
\usepackage[utf8]{inputenc}
\usepackage[spanish]{babel}
\usepackage{graphicx}
\usepackage{lmodern}

\title{Análisis Predictivo y Gestión de Datos}
\subtitle{Sesión 3: Exploración y Visualización de Patrones}
\author{Oscar Leonardo Rincón León}
\date{\today}

\begin{document}

\frame{\titlepage}

\begin{frame}{Objetivo de la sesión}
Analizar visualmente la estructura de los datos para guiar la selección de modelos predictivos.
\end{frame}

\begin{frame}{¿Por qué es importante la exploración de datos (EDA)?}
\begin{itemize}
    \item Permite comprender la estructura y comportamiento de los datos.
    \item Detecta errores, valores atípicos y relaciones inesperadas.
    \item Guía la selección del modelo predictivo y el preprocesamiento.
    \item Fortalece la interpretación de los resultados.
\end{itemize}
\end{frame}

\begin{frame}{Elementos clave en EDA}
\begin{itemize}
    \item \textbf{Distribuciones univariadas:} histogramas, boxplots, KDE.
    \item \textbf{Relaciones bivariadas:} scatterplots, boxplots por categoría.
    \item \textbf{Correlaciones:} mapas de calor.
    \item \textbf{Outliers y valores nulos:} detección visual y estadística.
    \item \textbf{Transformaciones:} log, normalización, escalado.
\end{itemize}
\end{frame}

\begin{frame}{Distribuciones y visualización univariada}
\begin{itemize}
    \item Ayudan a identificar asimetrías, colas largas, sesgo.
    \item La elección de bins en histogramas influye en la percepción.
    \item Boxplots muestran mediana, cuartiles y valores atípicos.
    \item KDE suaviza la distribución (estimación de densidad).
\end{itemize}
\end{frame}

\begin{frame}{Relaciones bivariadas y correlaciones}
\begin{itemize}
    \item \textbf{Diagramas de dispersión:} relaciones entre dos variables numéricas.
    \item \textbf{Mapas de calor:} matriz de correlación visual.
    \item \textbf{Boxplots por grupo:} para comparar variables numéricas entre categorías.
    \item Correlación $\neq$ causalidad, pero puede guiar hipótesis.
\end{itemize}
\end{frame}

\begin{frame}{Outliers y estructura de los datos}
\begin{itemize}
    \item Valores extremos pueden ser errores, casos reales o mal codificados.
    \item Su detección es clave para evitar distorsión en modelos.
    \item La visualización facilita su identificación antes del análisis.
    \item Algunos algoritmos son sensibles a outliers (ej. KNN, regresión).
\end{itemize}
\end{frame}

\begin{frame}{Visualización efectiva: más que estética}
\begin{itemize}
    \item El objetivo no es embellecer, sino informar y guiar decisiones.
    \item Cada gráfico debe responder a una pregunta específica.
    \item Buenas prácticas: claridad, etiquetas, escala apropiada.
    \item La visualización también comunica hallazgos al público no técnico.
\end{itemize}
\end{frame}

\begin{frame}{Herramientas prácticas en Python}
\begin{itemize}
    \item \texttt{matplotlib}: base para gráficos personalizados.
    \item \texttt{seaborn}: enfoque estadístico y estético para gráficos rápidos.
    \item \texttt{pandas}: soporte básico para gráficos integrados en DataFrames.
\end{itemize}
\end{frame}

\begin{frame}{Actividad práctica}
\begin{itemize}
    \item Usar el dataset limpio con 1300 registros.
    \item Visualizar:
    \begin{itemize}
        \item Histogramas y boxplots univariados.
        \item Diagramas de dispersión y mapas de calor.
    \end{itemize}
    \item Detectar patrones visuales útiles para la predicción.
    \item Reflexionar sobre posibles transformaciones.
\end{itemize}
\end{frame}

\begin{frame}{Cierre de la sesión}
\begin{itemize}
    \item El EDA no es opcional: mejora la comprensión y orienta decisiones.
    \item Una visualización bien hecha permite anticipar errores, confirmar hipótesis o replantear modelos.
    \item El análisis visual es tanto exploratorio como comunicativo.
\end{itemize}
\end{frame}

\end{document}
